%% bare_conf.tex
%% V1.4b
%% 2015/08/26
%% by Michael Shell
%% See:
%% http://www.michaelshell.org/
%% for current contact information.
%%
%% This is a skeleton file demonstrating the use of IEEEtran.cls
%% (requires IEEEtran.cls version 1.8b or later) with an IEEE
%% conference paper.
%%
%% Support sites:
%% http://www.michaelshell.org/tex/ieeetran/
%% http://www.ctan.org/pkg/ieeetran
%% and
%% http://www.ieee.org/

%%*************************************************************************
%% Legal Notice:
%% This code is offered as-is without any warranty either expressed or
%% implied; without even the implied warranty of MERCHANTABILITY or
%% FITNESS FOR A PARTICULAR PURPOSE! 
%% User assumes all risk.
%% In no event shall the IEEE or any contributor to this code be liable for
%% any damages or losses, including, but not limited to, incidental,
%% consequential, or any other damages, resulting from the use or misuse
%% of any information contained here.
%%
%% All comments are the opinions of their respective authors and are not
%% necessarily endorsed by the IEEE.
%%
%% This work is distributed under the LaTeX Project Public License (LPPL)
%% ( http://www.latex-project.org/ ) version 1.3, and may be freely used,
%% distributed and modified. A copy of the LPPL, version 1.3, is included
%% in the base LaTeX documentation of all distributions of LaTeX released
%% 2003/12/01 or later.
%% Retain all contribution notices and credits.
%% ** Modified files should be clearly indicated as such, including  **
%% ** renaming them and changing author support contact information. **
%%*************************************************************************


% *** Authors should verify (and, if needed, correct) their LaTeX system  ***
% *** with the testflow diagnostic prior to trusting their LaTeX platform ***
% *** with production work. The IEEE's font choices and paper sizes can   ***
% *** trigger bugs that do not appear when using other class files.       ***                          ***
% The testflow support page is at:
% http://www.michaelshell.org/tex/testflow/



\documentclass[a4, conference]{IEEEtran}
\IEEEoverridecommandlockouts
% \usepackage{graphicx}
% \usepackage{cite}
% \usepackage{amsmath,amssymb,amsfonts}
% \usepackage{algorithmic}

\def\BibTeX{{\rm B\kern-.05em{\sc i\kern-.025em b}\kern-.08em
    T\kern-.1667em\lower.7ex\hbox{E}\kern-.125emX}}

%make sure in A4 paper
\usepackage[left=1.57cm,right=1.57cm,top=0.95cm,bottom=2.54cm]{geometry}

% correct bad hyphenation here
\hyphenation{op-tical net-works semi-conduc-tor}


\begin{document}
%
% paper title
% Titles are generally capitalized except for words such as a, an, and, as,
% at, but, by, for, in, nor, of, on, or, the, to and up, which are usually
% not capitalized unless they are the first or last word of the title.
% Linebreaks \\ can be used within to get better formatting as desired.
% Do not put math or special symbols in the title.
\title{Bare Demo of IEEEtran.cls\\ for IEEE Conferences}

%\author{\small *No authors informations during review process.}
% author names and affiliations
% use a multiple column layout for up to three different
% affiliations
\author{\IEEEauthorblockN{Michael Shell}
    \IEEEauthorblockA{School of Electrical and\\Computer Engineering\\
        Georgia Institute of Technology\\
        Atlanta, Georgia 30332--0250\\
        Email: http://www.michaelshell.org/contact.html}
    \and
    \IEEEauthorblockN{Homer Simpson}
    \IEEEauthorblockA{Twentieth Century Fox\\
        Springfield, USA\\
        Email: homer@thesimpsons.com}
    \and
    \IEEEauthorblockN{James Kirk\\ and Montgomery Scott}
    \IEEEauthorblockA{Starfleet Academy\\
        San Francisco, California 96678--2391\\
        Telephone: (800) 555--1212\\
        Fax: (888) 555--1212}}

% conference papers do not typically use \thanks and this command
% is locked out in conference mode. If really needed, such as for
% the acknowledgment of grants, issue a \IEEEoverridecommandlockouts
% after \documentclass

% for over three affiliations, or if they all won't fit within the width
% of the page, use this alternative format:
% 
%\author{\IEEEauthorblockN{Michael Shell\IEEEauthorrefmark{1},
%Homer Simpson\IEEEauthorrefmark{2},
%James Kirk\IEEEauthorrefmark{3}, 
%Montgomery Scott\IEEEauthorrefmark{3} and
%Eldon Tyrell\IEEEauthorrefmark{4}}
%\IEEEauthorblockA{\IEEEauthorrefmark{1}School of Electrical and Computer Engineering\\
%Georgia Institute of Technology,
%Atlanta, Georgia 30332--0250\\ Email: see http://www.michaelshell.org/contact.html}
%\IEEEauthorblockA{\IEEEauthorrefmark{2}Twentieth Century Fox, Springfield, USA\\
%Email: homer@thesimpsons.com}
%\IEEEauthorblockA{\IEEEauthorrefmark{3}Starfleet Academy, San Francisco, California 96678-2391\\
%Telephone: (800) 555--1212, Fax: (888) 555--1212}
%\IEEEauthorblockA{\IEEEauthorrefmark{4}Tyrell Inc., 123 Replicant Street, Los Angeles, California 90210--4321}}




% use for special paper notices
%\IEEEspecialpapernotice{(Invited Paper)}




% make the title area
\maketitle

% As a general rule, do not put math, special symbols or citations
% in the abstract
\begin{abstract}
    The abstract goes here.
\end{abstract}

% no keywords




% For peer review papers, you can put extra information on the cover
% page as needed:
% \ifCLASSOPTIONpeerreview
% \begin{center} \bfseries EDICS Category: 3-BBND \end{center}
% \fi
%
% For peerreview papers, this IEEEtran command inserts a page break and
% creates the second title. It will be ignored for other modes.
\IEEEpeerreviewmaketitle



\section{Introduction}

The increasing volume of information within cultural heritage institutions, such as museum archives, presents significant challenges for effective knowledge management and access. Knowledge graphs (KGs) offer a powerful solution by representing entities and their relationships in a structured and machine-readable format, enabling advanced functionalities like semantic search, data integration, and reasoning. In the cultural heritage domain, ontologies such as the CIDOC Conceptual Reference Model (CRM) [cite CIDOC CRM paper] provide a standardized framework for describing cultural objects, events, and their interrelationships, facilitating interoperability and data sharing across institutions.

Traditionally, the construction of ontology-conformant KGs is a labor-intensive process requiring expert knowledge in both the domain and the ontology. However, the recent advancements in Large Language Models (LLMs) have demonstrated their remarkable capabilities in understanding and generating human language. Notably, LLMs exhibit strong in-context learning abilities, allowing them to generalize over a wide range of language tasks, including complex information extraction, without the need for expensive fine-tuning. This opens up new possibilities for automated knowledge extraction and KG construction, and in this case specifically, processing unstructured textual data from museum archives and mapping the extracted information to the concepts and relationships defined in ontologies like the CIDOC-CRM.

This paper investigates the feasibility and effectiveness of using state-of-the-art LLMs to construct ontology-conformant knowledge graphs from museum archives, specifically focusing on the CIDOC-CRM. We present a case study using the online archives of the Museum Nasional in Jakarta, Indonesia. We construct a textual dataset from web pages dedicated to specific collections or exhibits, then employ an in-context learning approach to extract knowledge graph triplets according to a CIDOC-CRM-derived ontology. A key aspect of our approach is the integration of this ontology directly within a single-stage prompting process, with our hypothesis being knowledge graphs that are semantically richer and more accurate.

To evaluate the constructed knowledge graphs, we employ both intrinsic and extrinsic evaluation methods. The intrinsic evaluation measures the conformance of the generated graphs to the CIDOC-CRM ontology. The extrinsic evaluation is conducted through a Knowledge Base Question Answering (KBQA) task, where we use the knowledge graph as context for an LLM to answer questions related to the museum collections. This study aims to contribute to the growing body of research on leveraging LLMs and ontologies for knowledge graph generation, specifically in the cultural heritage domain, with the ultimate motivation of improving museum data management and access.

\section{Related Work}

\subsection{Cultural Heritage Ontologies and Semantic Web}

In computer science, ontologies serve as formal, explicit specifications of a shared conceptualization for a domain of interest. They define the relevant classes (concepts), properties (attributes), and relationships between entities, along with axioms constraining their interpretation. Ontologies provide the semantic schema or backbone for Knowledge Graphs (KGs), which represent entities and their connections as nodes and edges. The common vocabulary and structure of ontologies ensures consistency, enables the integration of heterogeneous data, and facilitates semantic querying and logical reasoning over the interconnected information within a KG.

The cultural heritage (CH) domain, characterized by its diverse institutions (museums, archives, libraries), varied collections, and heterogeneous documentation practices, faces significant challenges in data integration and interoperability. Ontologies play a critical role in addressing these issues by providing a common semantic framework. They act as a mediating layer, enabling disparate CH information sources—originally described using various metadata standards or local schemas—to be mapped, linked, and queried based on shared meaning, thereby facilitating integrated access and interdisciplinary research.

The CIDOC Conceptual Reference Model (CIDOC-CRM, ISO 21127) is the preeminent standard ontology designed specifically for the integration of cultural heritage information. Developed over decades and grounded in the analysis of documentation practices across museums, libraries, and archives, CIDOC-CRM provides a formal, extensible semantic framework for describing the complex concepts and relationships inherent in CH data. Its event-centric modeling paradigm, which structures information around events linking actors, objects, places, and times, is particularly adept at capturing the rich contextual history crucial for understanding cultural objects and collections. Consequently, CIDOC-CRM serves as a vital tool for structuring knowledge extracted from diverse sources, including museum archives, and building interoperable cultural heritage knowledge graphs.

\subsection{Ontology-Guided Knowledge Graph Generation}

Ontology-guided or ontology-driven knowledge graph generation focuses on leveraging formal ontologies to guide the extraction of structured knowledge from unstructured or semi-structured text. By providing a predefined schema of concepts and relationships, ontologies ensure that the generated knowledge graphs are semantically consistent, well-structured, and aligned with a specific domain [Construction of Knowledge Graphs: Current State and Challenges]. This approach is particularly relevant in domains like cultural heritage, where established ontologies like the CIDOC Conceptual Reference Model (CRM) offer a standardized framework for representing cultural objects, events, and their relationships.

The Text2KGBench benchmark [Text2KGBench: A Benchmark for Ontology-Driven Knowledge Graph Generation from Text] provides a comprehensive evaluation framework for ontology-driven knowledge graph generation. It uses an established ground truth to assess the quality of extracted knowledge graphs based on accuracy, ontology conformance and hallucination. Research utilizing this benchmark, such as the development of the specialized language model T5 [Domain Ontology-Driven Knowledge Graph Generation from Text], underscores the ongoing efforts to improve the accuracy and reliability of ontology-driven KG generation using advanced language models. These studies often experiment with different prompting strategies and model architectures to optimize the extraction process.

Several recent studies have explored the use of Large Language Models (LLMs) for knowledge graph generation. \cite{xu2024chattf} demonstrates an end-to-end intelligent question answering system for Chinese traditional folklore. Their approach utilizes a two-stage prompting method guided by an ontology derived from CIDOC-CRM to extract knowledge graph triplets from scraped web content about Chinese traditional folklore. [Transforming Indonesian Geography Education Books Into Knowledge Graphs Using ChatGPT LLMs] further illustrates the effectiveness of LLMs in extracting knowledge graphs, though this case used an ontology-free method. These examples highlight the potential of LLMs to construct ontology-conformant knowledge graphs in their respective domains.

Furthermore [Assessing LLMs suitability for knowledge graph completion] investigates the suitability of LLMs for knowledge graph completion tasks, demonstrating that well-crafted prompts, particularly those employing in-context learning and chain-of-thought techniques, can yield significant improvements. This highlights the importance of effective prompting methodologies in guiding LLMs to generate accurate and consistent knowledge graphs aligned with a given ontology.

The potential of LLMs to understand and classify terms according to established ontologies like CIDOC CRM has also been shown by [Leveraging Large Language Models for Classification of Cultural Heritage Domain Terms: A Case Study on CIDOC CRM].

Our research focuses on the extraction of knowledge graphs from museum archives using an ontology derived from the CIDOC-CRM. Unlike the two-stage prompting approach employed by [ChatTf], we explore the effectiveness of a single-stage prompting method. Moreover, our study incorporates a comprehensive evaluation strategy that includes both intrinsic (ontology conformance) and extrinsic (KBQA task) metrics.

\section{Methods}

The methodology is structured around the following key steps: ontology and dataset construction, knowledge graph generation (ontology-driven and ontology-free), and knowledge graph evaluation (intrinsic and extrinsic).

\subsection{Ontology Construction}

This study employs the CIDOC Conceptual Reference Model (CRM) [cite CIDOC-CRM standard] to structure the knowledge graphs extracted from cultural heritage texts. We derived a smaller ontology for use within our LLM prompting framework, specialized to our specific data source: Museum Nasional web pages describing collection items. While the primary reason was to save token costs, as the ontology must be supplied in each extraction prompt, this aligns with typical implementations where the full CRM is specialized for particular applications [cite].

Our derived ontology profile was constructed by identifying the CIDOC-CRM classes and properties most relevant to describing museum objects, their history, and cultural context based on the expected content of the source material. The selection included core entities for physical objects, actors, places, and time-spans, key events related to object lifecycle and provenance (like production, acquisition, modification), and essential descriptive concepts (such as materials, types, titles, and dimensions).

\subsection{Dataset Construction}

The museum's official public website contains a wealth of information about its collection, with many exhibits and artifacts having dedicated articles. These articles were used to construct a dataset of textual information from the museum's collection, with each entry representing a complete article dedicated to a particular item.

A custom Question-Answering (QA) dataset was also created to test knowledge about the information contained within the museum archives and to evaluate whether a knowledge graph contains the expected information from the source text. The QA dataset includes multiple-choice questions that assess both individual item-specific facts as well as comparative facts across multiple items.

\subsection{Knowledge Graph Generation}

We implemented two distinct approaches for generating knowledge graphs from the constructed dataset: an ontology-driven approach and a ontology-free approach. Both approaches utilized a single-stage prompting method.

\subsubsection{Ontology-Guided}

This approach leverages an ontology to guide the LLM in extracting structured information. The process consists of the following steps:

\begin{enumerate}
    \item Ontology Verbalization: The ontology is verbalized into a natural language description.
    \item System Prompt Construction: A system prompt is constructed from the following elements
          \begin{itemize}
              \item A clear task description instructing the LLM to extract entities and relationships from the input text and represent them as a knowledge graph adhering to the provided ontology.
              \item Example input-output pairs demonstrating the desired output format.
              \item The verbalized ontology, appended at the end of the system prompt.
          \end{itemize}
    \item Chat History Initialization: The chat history is initialized with an example input and output to serve as an in-context learning mechanism.
    \item Triplet Extraction: The system prompt, chat history and text chunk are sent to the LLM. The response is parsed as a list of triplets and serialized as a JSONL file where each line contains the triplets extracted from a chunk.
\end{enumerate}

\subsubsection{Ontology-Free}

The ontology-free approach aims to extract general knowledge graphs without specific guidance from the CIDOC-CRM ontology. It is nearly identical to the ontology-driven approach with the following key differences:

\begin{itemize}
    \item The task description instructed the LLM to identify entities and relationships relevant to museum collections and exhibits in the input text.
    \item The example input-output pairs demonstrated the extraction of general relationships without mapping them to a specific ontology.
    \item No verbalized ontology was included in the system prompt.
\end{itemize}

This ontology-free approach serves as a comparison point within the extrinsic evaluation, specifically to measure the impact of incorporating the ontology on KBQA performance, and indirectly, the accuracy of the knowledge graph.

\bibliographystyle{IEEEtran}
\bibliography{IEEEabrv,References}

% that's all folks
\end{document}
